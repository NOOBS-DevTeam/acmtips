\chapter{Строки}
\section{Алгоритмы, которых нет на e-maxx}
\subsection{Наибольшая общая подпоследовательность $|S_1| \cdot |S_2|$}
Задача решается ДП. Состояние будет характеризовать два параметра: длина первой и второй строки. Изначально $dp(n_1,n_2) = 0$. Переход к следующему состоянию \newline
$dp(n_1, n_2) = \left \{ 
\begin{array}{ll}
 0, n_1 = 0\ or\ n_2 = 0 \\
 dp(n_1 - 1, n_2 - 1) + 1,  s[n_1] = s[n_2] \\
 max(dp(n_1 - 1, n_2), dp(n_1, n_2 - 1)),  s[n_1] \neq s[n_2]
 \end{array}
 \right.$
 Чтобы востановить ответ нужно запоминать из какого состояния мы попали в текущее, а в конце пройти из ячейки $dp(|S_1|,|S_2|)$ до $dp(0,0)$ по этому маршруту.
 \subsection{Наибольшая общая подстрока}
 Первой строкой входного файла следует цифра обозначающая количество строк в которых следует искать подстроку. Строки для поиска следуют дальше.
\begin{lstlisting}[language = Python]
import sys
list = sys.argv[1:]
fileName = list[0]
data = open(fileName).read().splitlines()
numOfStrings = int(data.pop(0))-1
data = filter(None, data)
data.sort(key=len)
leastStr = data.pop(0)
maxSharedStr = ''
while len(leastStr) > len(maxSharedStr):
    robTestStr = leastStr
    while len(robTestStr) > len(maxSharedStr):
        numOfConcidence = 0
        for compatStr in data:
            if robTestStr in compatStr:
                numOfConcidence += 1
            else:
                break
        if numOfConcidence == numOfStrings and len(robTestStr) > len(maxSharedStr):
            maxSharedStr = robTestStr
        robTestStr = robTestStr[:-1]
    leastStr = leastStr[1:]
print maxSharedStr
sys.exit(0)
\end{lstlisting}
