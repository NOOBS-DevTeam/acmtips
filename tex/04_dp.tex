\chapter{ДП}
\section{Задача о рюкзаке}
\subsection{Класический рюкзак О($N*M$)}
\begin{lstlisting}[language = C++]
for i = 0..W
  A[0][i] = 0;
for i = 0..N
  A[i][0] = 0;   //Первые элементы приравниваем к 0
for k = 1..N               
  for s = 0..W   //Перебираем для каждого k все вместимости 
    if s >= w[k]    //Если текущий предмет вмещается в рюкзак
      A[k][s] = max(A[k-1][s], A[k-1][s-w[k]]+p[k]); //выбираем класть его или нет
    else 
      A[k][s] = A[k-1][s];             //иначе, не кладем
//Затем найдем набор ans предметов, входящих в рюкзак, рекурсивной функцией:
findAns(k, s)
  if A[k][s] == 0 
    return;
  if A[k-1][s] == A[k][s]
    findAns(k-1, s);
  else 
    findAns(k-1, s - w[k]);
    ans.push(k);
\end{lstlisting}
\subsection{Ограниченный рюкзак О($N*M^2$)}
Тут каждого предмета определенное кол-во.\newline
Пусть d(i,c) максимальная стоимость любого возможного числа предметов типов от 1 до i, суммарным весом до c.
Заполним d(0,c) нулями.
Тогда меняя i от 1 до N, рассчитаем на каждом шаге d(i,c), для c от 0 до W, по рекуррентной формуле:
d(i,c) = max(d(i - 1, c - lw_i) + lp_i) по всем целым l из промежутка 0 \le l \le min(b_i,\lfloor c/w_i \rfloor).
Если не нужно восстанавливать ответ, то можно использовать одномерный массив d(c) вместо двумерного.
После выполнения в d(N,W) будет лежать максимальная стоимость предметов, помещающихся в рюкзак.
\begin{lstlisting}[language = C++]
for i = 0..W // база
  d[0][i] = 0;
for i = 1..N             
  for c = 0..W   //Перебираем для каждого i, все вместимости 
    d[i][c] = d[i - 1][c];
    for l = min(b[i], c / w[i]) .. 1 // ищем l для которого выполняется максимум
        d[i][c] =  max(d[i][c], d[i - 1][c - l * w[i]] + p[i] * l);
\end{lstlisting}
\subsection{Неограниченный рюкзак О($N*M$)}
Тут каждого предмета бесконечно.\newline
Пусть d(i,c) максимальная стоимость любого количества вещей типов от 1 до i, суммарным весом до c включительно.
Заполним d(0,c) нулями.
Тогда меняя i от 1 до N, рассчитаем на каждом шаге d(i,c), для c от 0 до W, по рекуррентной формуле:\newline
d(i,c) = \begin{cases}  d(i - 1, c) & for\ c = 0, ..., w_i - 1; \\  max(d(i - 1, c), d(i, c - w_i) + p_i) & for\ c = w_i, ..., W;  \end{cases}
После выполнения в d(N,W) будет лежать максимальная стоимость предметов, помещающихся в рюкзак.
Если не нужно восстанавливать ответ, то можно использовать одномерный массив d(c) вместо двумерного и использовать формулу:
d(c) = max(d(c), d(c - w_i) + p_i);
\subsection{Непрерывный рюкзак О($N*log(N)$)}
Тут можно дробить предметы.\newline Можно решить жадиной.
\begin{lstlisting}[language = C++]
sort(); // сортируем в порядке убывания удельной стоимости.
for i = 1..N // идем по предметам            
      if W >= w[i]    //если помещается - берем
       sum += p[i];
       W -= w[i];
   else
       sum += W / w[i] * p[i];// иначе берем сколько можно и выходим
       break;
\end{lstlisting}
\subsection{Задача о сумме подмножеств О($N*M$)}
Тут стоимость предмета равна весу.\newline
Пусть d(i,c) максимальная сумма  \le c, подмножества взятого из 1, ...,\ i элементов.
Заполним d(0,c) нулями.
Тогда меняя i от 1 до N, рассчитаем на каждом шаге d(i,c), для c от 0 до W, по рекуррентной формуле:\newline
d(i,c) = \begin{cases}  d(i - 1, c) & for\ c = 0, ..., w_i - 1; \\  max(d(i - 1, c), d(i - 1, c - w_i) + w_i) & for\ c = w_i, ..., W;  \end{cases}
После выполнения в d(N,W) будет лежать максимальная сумма подмножества, не превышающая заданное значение.
\subsection{Задача о размене О($N*M$)}
Тут всех предметов бесконечно, но нужно набрать вес ровно W наименьшим кол-вом предметов.\newline
Пусть d(i,c) минимальное число предметов, типов от 1 до i, необходимое, чтобы заполнить рюкзак вместимостью c.
Пусть d(0,0) = 0, а d(0,c) = \inf для всех c > 0.
Тогда меняя i от 1 до N, рассчитаем на каждом шаге d(i,c), для c от 0 до W, по рекуррентной формуле:\newline
d(i,c) = \begin{cases}  d(i - 1, c) & for\ c = 0, ..., w_i - 1; \\  min(d(i - 1, c), d(i, c - w_i) + 1) & for\ c = w_i, ..., W;  \end{cases}
После выполнения в d(N,W) будет лежать максимальная стоимость предметов, помещающихся в рюкзак.
Если не нужно восстанавливать ответ, то можно использовать одномерный массив d(c) вместо двумерного и использовать формулу:
d(c) = min(d(c), d(c - w_i) + 1) \qquad  for\ c = w_i, ..., W.
\subsection{Задача о сумме подмножеств О($N*M$)}
Тут стоимость предмета равна весу.\newline
