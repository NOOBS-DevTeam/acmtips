\chapter{Разбор выражений (нисходящий парсер)}
Дано выражение. Начинаем парсить его функцией E1, которая обрабатывает самые низкоприоритетные операции (в нашем случае '+', '-').
\begin{lstlisting}[language=Python]
def E1():
	res = E2()
	while True:
		c = getc()
		if c == '+':
			res += E2()
		elif c == '-':
			res -= E2()
		else:
			putc(c)
			return res
\end{lstlisting}
Сначала происходит обработка атома, стоящего слева от знака ('+', '-'), затем обработка каждого атома между знаками. Обработку этих атомов производит функция $E2()$. Это функция более низкого ранга, которая обрабатывает более приоритетные операции (в нашем случае '*' и '/').
\begin{lstlisting}[language=Python]
def E2():
	res = E3()
	while True:
		c = getc()
		if c == '*':
			res *= E3()
		elif c == '/':
			res /= E3()
		else:
			putc(c)
			return res
\end{lstlisting}
Функция работает аналогично $E1()$. Таким образом мы можем поддерживать сколько угодно операций различных приоритетов, добавляя функции.

Перейдем к обработке скобок:
\begin{lstlisting}[language=Python]
def E3():
	if c == '(':
		res = E1()
		c = getc()
		return res
	else:
		putc(c)
		return E4()
\end{lstlisting}
Берём символ, если он скобка, тогда обрабатываем выражение внутри и считываем закрывающую скобку.

Теперь рассмотрим считывание числа. Ничего сложного:
\begin{lstlisting}[language=Python]
def E4():
	res = 0
	while True:
		c = getc()
		if str.isdigit(c):
			res = res * 10 + int(c)
		else:
			putc(c)
			return res
\end{lstlisting}
