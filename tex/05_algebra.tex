\chapter{Алгебра}
\section{Алгоритм Евклида}
\subsection{НОД, НОК (gcd, lcm)}
$
	gcd(a, b) = 
		\begin{cases}
			a &\text{если } a = 0 \\
			b & \text{иначе}
		\end{cases}
$
\newline
$lcm(a, b) = \frac{a \cdot b}{gcd(a, b)}$

\subsection{Расширенный алгоритм Евклида}
\begin{lstlisting}[language=C++]
int gcdex (int a, int b, int & x, int & y) {
	if (a == 0) {
		x = 0; y = 1;
		return b;
	}
	int x1, y1;
	int d = gcd (b%a, a, x1, y1);
	x = y1 - (b / a) * x1;
	y = x1;
	return d;
}
\end{lstlisting}
Функция возвращает нод и коеффициенты $x$, $y$ по сслыкам

\subsection{Восстановление в кольце по модулю}
\begin{lstlisting}[language=C++]
int x, y;
int g = gcdex (a, m, x, y);
if (g != 1)
	cout << "no solution";
else {
	x = (x % m + m) % m;
	cout << x;
}
\end{lstlisting}

\section{Решето Эратосфена}
\subsection{Классический вариант}
Дано число n. Требуется найти все простые в отрезке $[2; n]$.
Решето Эратосфена решает эту задачу за $O (n \log \log n)$ \newline
Запишем ряд чисел $1 \ldots n$, и будем вычеркивать сначала все числа, делящиеся на $2$, кроме самого числа $2$, затем деляющиеся на $3$, кроме самого числа $3$, затем на $5$ и так далее $\ldots$
\begin{lstlisting}[language=C++]
int n;
vector<char> prime (n+1, true);
prime[0] = prime[1] = false;
for (int i=2; i<=n; ++i)
	if (prime[i])
		if (i * 1ll * i <= n)
			for (int j=i*i; j<=n; j+=i)
				prime[j] = false;
\end{lstlisting}

\subsection{Линейное решето}
Чуть быстрее по времени, чем классическое $O (N)$. Цена вопроса - оверхед по памяти.\
Пусть $lp[i]$ -- минимальный простой делитель числа $i, 2\leq i\leq n$. \
\begin{itemize}
\item $lp[i] = 0$ -- число $i$ -- простое
\item $lp[i] \neq 0$ -- число $i$ -- составное
\end{itemize}
\begin{lstlisting}[language=C++]
const int N = 10000000;
int lp[N+1];
vector<int> pr;
 
for (int i=2; i<=N; ++i) {
	if (lp[i] == 0) {
		lp[i] = i;
		pr.push_back (i);
	}
	for (int j=0; j<(int)pr.size() && pr[j]<=lp[i] && i*pr[j]<=N; ++j)
		lp[i * pr[j]] = pr[j];
}
\end{lstlisting}
Вектор $lp[\ ]$ можно как-то использовать для факторизации чисел

\section{Разбор выражений}
Дано выражение. Начинаем парсить его функцией E1, которая обрабатывает самые низкоприоритетные операции (в нашем случае '+', '-').
\begin{lstlisting}[language=Python]
def E1():
	res = E2()
	while True:
		c = getc()
		if c == '+':
			res += E2()
		elif c == '-':
			res -= E2()
		else:
			putc(c)
			return res
\end{lstlisting}
Сначала происходит обработка атома, стоящего слева от знака ('+', '-'), затем обработка каждого атома между знаками. Обработку этих атомов производит функция $E2()$. Это функция более низкого ранга, которая обрабатывает более приоритетные операции (в нашем случае '*' и '/').
\begin{lstlisting}[language=Python]
def E2():
	res = E3()
	while True:
		c = getc()
		if c == '*':
			res *= E3()
		elif c == '/':
			res /= E3()
		else:
			putc(c)
			return res
\end{lstlisting}
Функция работает аналогично $E1()$. Таким образом мы можем поддерживать сколько угодно операций различных приоритетов, добавляя функции.

Перейдем к обработке скобок:
\begin{lstlisting}[language=Python]
def E3():
	if c == '(':
		res = E1()
		c = getc()
		return res
	else:
		putc(c)
		return E4()
\end{lstlisting}
Берём символ, если он скобка, тогда обрабатываем выражение внутри и считываем закрывающую скобку.

Теперь рассмотрим считывание числа. Ничего сложного:
\begin{lstlisting}[language=Python]
def E4():
	res = 0
	while True:
		c = getc()
		if str.isdigit(c):
			res = res * 10 + int(c)
		else:
			putc(c)
			return res
\end{lstlisting}
