\chapter{Графы}
\section{Основы}
Граф - множество вершин и ребер(заданных явно или не явно). Понятия используемые в дальнейшем:
\begin{itemize} 
\item 
    Ациклический граф
    \begin{mydef}
        Граф без циклов
    \end{mydef}
\item     
    Влентность/Степень вершины
    \begin{mydef}
        Количество ребер входящих/выходящих в вершину
    \end{mydef}
\item     
    Взвешенный граф
    \begin{mydef}
        Граф в котором у каждого ребра есть стоимость
    \end{mydef}
\item     
    Висячая вершина
    \begin{mydef}
        Вершина со степенью один
    \end{mydef}
\item     
    Гамильтонов путь
    \begin{mydef}
        Путь в графе содержащий каждую вершину ровно один раз
    \end{mydef}
\item     
    Гамильтонов цикл
    \begin{mydef}
        Цикл содержащий каждую вершину ровно один раз
    \end{mydef}
\item     
    Двудольный граф
    \begin{mydef}
        Граф в котором можно выделить два множества такие, что между         любыми двумя вершинами одного множества нет ребер.
    \end{mydef}
\item     
    Компонента связности
    \begin{mydef}
        Множество вершин и ребер графа такое, что из каждой его вершины
        достижима любая другая вершина этого множества
    \end{mydef}
\item     
    Компонента сильной связности
    \begin{mydef}
        Множество вершин и ребер ориентированного графа такое, что из каждой его           вершины достижима любая другая вершина этого множества
    \end{mydef}
\item     
    Кратные ребра
    \begin{mydef}
        Ребра связывающие одну и ту же пару вершин
    \end{mydef}
\item     
    Минимальный каркас
    \begin{mydef}
        Множество ребер соеденяющих все вершины графа без циклов и имеющее               минимальный суммарный вес
    \end{mydef}
\item     
    Паросочетания
    \begin{mydef}
        Множество попарно не смежных ребер
    \end{mydef}
\item     
    Точка сочленения
    \begin{mydef}
        Вершина после удаления которой количество компонент связности возрастает
    \end{mydef}
\item     
    Эйлеров путь
    \begin{mydef}
        Путь в графе содержащий каждое ребро ровно один раз
    \end{mydef}
\item     
    Эйлеров цикл
    \begin{mydef}
        Цикл содержащий каждое ребро ровно один раз
    \end{mydef}
\end{itemize}
\section{Идентификация графа}
\subsection{Поиск цикла и проверка на ацикличность}
Воспользуемся поиском в глубину. Окрасим все вершины в белый. Запускаясь от вершины перекрашиваем ее в серый, а выходя из нее красим в черный. Если ${dfs}$ попытается пойти в серую вершину, значит мы нашли цикл, который сможем вывести с помощью массива предков, иначе граф циклов не имеет. Далее реализация на списках смежности.
\begin{lstlisting}[language=Python]
bool dfs (int v) {
	cl[v] = 1; //colors
	for (int i = 0; i < g[v].size(); i++) {
		int to = g[v][i];
		if (cl[to] == 0) {
			p[to] = v; 
			if (dfs(to)) //if son have cycle then parent have cycle
			    return true;
		}
		else if (cl[to] == 1) {
			cycle_end = v;
			cycle_st = to;
			return true;
		}
	}
	cl[v] = 2;
	return false;
}
\end{lstlisting}
\subsection{Гамильтонов граф}
Пусть ${n}$ количесттво вершин графа, а ${\delta}$ минимальная степень вершины в графе, тогда граф имеет гамильтонов цикл если ${n\geq 3}$ и ${\delta \geq \frac{n}{2}}$
\subsection{Двудольный граф O($n$)}
Так как граф является двудольным тогда и только тогда, когда все циклы четны, определить двудольность можно за один проход в глубину. На каждом шаге обхода в глубину помечаем вершину. Допустим мы пошли в первую вершину — помечаем её как 1. Затем просматриваем все смежные вершины и если не помечена вершина, то на ней ставим пометку 2 и рекурсивно переходим в нее. Если же она помечена и на ней стоит та же пометка, что и у той, из которой шли (в нашем случае 1), значит граф не двудольный.
\begin{lstlisting}[language=C++]
bool dfs (int v, int c) {
	cl[v] = c; //colors
	for (int i = 0; i < g[v].size(); i++) {
		int to = g[v][i];
		if (cl[to] == 0) {
		    return dfs(to, max(1, (c + 1) % 2));
		}
		else 
		    return cl[to] != c;
	}
}
\end{lstlisting}
\subsection{Компоненты связности O($n$)}
Поиск компонент свзязности можно осуществить многими методами, в том числе и поиском в глубину. Окрасим все вершины в индивидуальные цвета. Запуская $[dfs]$ от каждой вершины будем перекрашивать в ее цвет все вершины в этой компоненте. В конце алгоритма у нас останется столько цветов, сколько компонент связности в графе, а цвет каждой вершины будет идентифицировать ее компоненту.
\begin{lstlisting}[language=C++]
void dfs (int v, int c) {
    if (!used[v])
        return;
	cl[v] = c; //colors
	for (int i = 0; i < g[v].size(); i++) {
		int to = g[v][i];
		if (cl[to]!= cl[v]) {
		    dfs(to, c);
		    return;
		}
}
int main(){
    ...
    for (int i = 0; i < N; i++){
        dfs(i, i);
    }
}
\end{lstlisting}
\subsection{Компоненты сильной связности O($n+m$)}
Решим эту задачу за несколько обходов в глубину. Сначала топологически (по времени выхода) отсортируем граф, затем обойдем транспонированый(инвертированый) граф в этом порядке. Найденые компоненты связности этого графа и будут компонентами сильной связности исходного графа.
\begin{lstlisting}[language=C++]
vector < vector<int> > g, gr; // граф и транспортированый граф
vector<char> used;
vector<int> order, component; // порядок топ сорта и компонента сильной связности
 
void dfs1 (int v) {
	used[v] = true;
	for (int i = 0; i < g[v].size(); i++)
		if (!used[g[v][i]])
			dfs1(g[v][i]);
	order.push_back(v);
}
 
void dfs2 (int v) {
	used[v] = true;
	component.push_back(v);
	for (int i = 0; i < gr[v].size(); i++)
		if (!used[gr[v][i]])
			dfs2(gr[v][i]);
}
 
int main() {
	int n, m;
	cin >> n >> m;
	for (int i = 0; i < m; i++){
		int a, b;
		cin >> a >> b;
		g[a].push_back(b);
		gr[b].push_back(a);
	}
 
	used.assign(n, false);
	for (int i = 0; i < n; i++)
		if (!used[i])
			dfs1(i);
	used.assign(n, false);
	for (int i = 0; i < n; i++) {
		int v = order[n - 1 - i];
		if (!used[v]) {
			dfs2 (v);
			//... вывод component ...
			component.clear();
		}
	}
}
\end{lstlisting}
\subsection{Минимальное остовное дерево O($M*log(M)$)}
Алгоритм Кр(а|у)скала. Составим список ребер, отсортируем их по возрастанию весов. Распределим все вершины в соответствующие им множества, для работы с ними воспользуемся СНМ. Будем добавлять ребра по порядку, но только, если они соединяют вершины из разных множеств. После этого объединим их множества.
\begin{lstlisting}[language=C++]
vector<int> p (n);//Множества

int dsu_get(int v) {
	return(v == p[v]) ? v : (p[v] = dsu_get(p[v]));
}

void dsu_unite(int a, int b) {
	a = dsu_get(a);
	b = dsu_get(b);
	if (rand() & 1)
		swap(a, b);
	if (a != b)
		p[a] = b;
}

int main(){
            .....
sort (g.begin(), g.end()); // список ребер <pair<вес, pair<начало, конец> > > 
p.resize (n);
for (int i = 0; i < n; i++)
	p[i] = i;
for (int i = 0; i < m; i++) {
	int a = g[i].second.first,  b = g[i].second.second,  l = g[i].first;
	if (dsu_get(a) != dsu_get(b)) {
		cost += l; // вес каркаса
		res.push_back(g[i].second); // каркас
		dsu_unite(a, b);
	}
}
\end{lstlisting}
\subsection{Точки сочленения}
Алгоритм для связного графа. Запустим поиск в глубину от произвольной вершины. Основная идея алгоритма: вершина является точка сочлинения, тогда и только тогда, когда невозможно попасть в предков этой вершины из её сыновей. Исключение составляет лишь вершина старта $dfs$, ведь у нее нет предков. Она будет точкой сочленения, если у нее больше одного сына, выевленного поиском в глубину
\begin{lstlisting}[language=C++]
vector<int> g[MAXN];
bool used[MAXN];
int timer, tin[MAXN], fup[MAXN];
 
void dfs (int v, int p = -1) {
	used[v] = true;
	tin[v] = fup[v] = timer++;
	int children = 0;
	for (int i = 0; i < g[v].size(); i++) {
		int to = g[v][i];
		if (to == p)  continue;
		if (used[to])
			fup[v] = min (fup[v], tin[to]);
		else {
			dfs (to, v);
			fup[v] = min (fup[v], fup[to]);
			if (fup[to] >= tin[v] && p != -1)
				IS_CUTPOINT(v);
			children++;
		}
	}
	if (p == -1 && children > 1)
		IS_CUTPOINT(v);
}
 
int main() {
	int n;
 
	timer = 0;
	for (int i = 0; i < n; i++)
		used[i] = false;
	dfs (0);
}
\end{lstlisting}
\subsection{Эйлеров путь/цикл}
Необходимым условием Эйлерово циклаи пути является наличие не более одной компоненты связности с ненулевым числом ребер. Если выполняется это условие и степень каждой вершины четна, то в графе существует Эйлеров цикл, а если степень всех вершин, кроме двух четна, то существует Эйлеров путь.Алгоритм напоминает поиск в глубину. Главное отличие состоит в том, что пройденными помечаются не вершины, а ребра графа. Начиная со стартовой вершины v строим путь, добавляя на каждом шаге не пройденное еще ребро, смежное с текущей вершиной. Вершины пути накапливаются в стеке S. Когда наступает такой момент, что для текущей вершины w все инцидентные ей ребра уже пройдены, записываем вершины из S в ответ, пока не встретим вершину, которой инцидентны не пройденные еще ребра. Далее продолжаем обход по не посещенным ребрам. Для поиска Эйлерова пути необходимо начинать с вершины у которой степень нечетна, и добавить ребро между вершинами с нечетными степенями, а из найденного цикла удалить добавленное ребро. Псевдокод:
\begin{lstlisting}[language=Python]
findPath(v):
   S.clear()
   S.add(v)
   while not S.isEmpty():
     w := S.top()
     if E contains(w, u):
       S.add(u)
       remove(w, u)
     else:
       S.pop()
       print w
\end{lstlisting}
\section{Кратчайший путь}
Все дальнейшие алгоритмы находят кратчайшие пути в одной компоненте связности при отсутствии циклов отрицательного веса в них, которые лекго обнаружить запустив алгоритм Форд-Беллмана один лишний раз, и если он обновит расстояние хоть до одной вершины, то в этой компоненте связности присутствует цикл отрицательного веса.
\subsection{Дейкстра}
Ищет путь от одной вершины до всех остальных. Каждой вершине из V сопоставим метку — минимальное известное расстояние от этой вершины до a. На каждом шаге он «посещает» одну вершину и пытается уменьшать метки. Работа алгоритма завершается, когда все вершины посещены.
\subsubsection{O($N^2$)}
\begin{lstlisting}[language=Python]
def Dijkstra(N, S, matrix):
	valid = [True]*N        
	weight = [1000000]*N
	weight[S] = 0
	for i in range(N):
		min_weight = 1000001
		ID_min_weight = -1
		for i in range(len(weight)):
			if valid[i] and weight[i] < min_weight:
				min_weight = weight[i]
				ID_min_weight = i
		for i in range(N):
			if weight[ID_min_weight] + matrix[ID_min_weight][i] < weight[i]:
				weight[i] = weight[ID_min_weight] + matrix[ID_min_weight][i]
		valid[ID_min_weight] = False
	return weight
\end{lstlisting}
\subsubsection{O($Mlog(N)$)}
\begin{lstlisting}[language=C++]
const int INF = 1000000000;
 
int main() {
	int n;
	...
	vector < vector < pair<int,int> > > g (n);
	...
	int s = ...; // стартовая вершина
 
	vector<int> d (n, INF),  p (n);
	d[s] = 0;
	set < pair<int,int> > q;
	q.insert (make_pair (d[s], s));
	while (!q.empty()) {
		int v = q.begin()->second;
		q.erase (q.begin());
 
		for (size_t j=0; j<g[v].size(); ++j) {
			int to = g[v][j].first,
				len = g[v][j].second;
			if (d[v] + len < d[to]) {
				q.erase (make_pair (d[to], to));
				d[to] = d[v] + len;
				p[to] = v;
				q.insert (make_pair (d[to], to));
			}
		}
	}
}
\end{lstlisting}
\subsection{Флой-Уоршелл O($N^3$)}
Пути между всеми парами вершин. Работает всегда.
\begin{lstlisting}[language=C++]
for k = 1 to n
  for i = 1 to n
    for j = 1 to n
      W[i][j] = min(W[i][j], W[i][k] + W[k][j])
\end{lstlisting}
\subsection{Форд-Беллман O($M*N$) }
Находит путь от одной вершины до всех. Работает всегда.
\begin{lstlisting}[language=C++]
struct edge {
	int a, b, cost;
};
 
int n, m, v;
vector<edge> e;
const int INF = 1000000000;
 
void solve() {
	vector<int> d (n, INF);
	d[v] = 0;
	for (int i=0; i<n-1; ++i)
		for (int j=0; j<m; ++j)
			if (d[e[j].a] < INF)
				d[e[j].b] = min (d[e[j].b], d[e[j].a] + e[j].cost);
	// вывод d, например, на экран
}
\end{lstlisting}
\section{Паросочетания}
\subsection{Наибольшее паросочетание O($N*M$)}
Введем новое понятие понятия:
\begin{itemize}
\item 
    Увеличивающая цепь 
    \begin{mydef}
        Простой путь, ребра которого поочередно входят/невходят в паросочетание, а первая и последняя вершина не пренадлежат паросочетанию
    \end{mydef}
\end{itemize}
\begin{theorem}Просочетание является максимальным тогда и только тогда, когда не существует увеличивающих его цепей
\end{theorem}
В основе алгоритма лежит эта теорема, позволяющая "улучшать" любое паросочетание.\newline
Будем считать, что граф уже разбит на две доли. Для каждой вершины из первой доли делаем следующее: если у текущей вершины уже есть ребро включенное в паросочетание, то переходим к следующей вершине, иначе запускаем поиск увеличивающей цепи из этой вершины. 
Поиск увеличивающей цепи. \newline
Для каждого ребра из этой вершины делаем следующее: пусть смежная с текущим ребром вершина не принадлежит паросочетанию, значит увеличивающая цепь найдена, добавим это ребро к паросочетанию, иначе ищем увеличивающую цепь из новой вершины. Максимальное паросочетание будет найдено после проверки всех вершин из первой доли. \newline
Т.к. на каждом шаге алгоритм улучшает текущее паросочетание, то асимптотику можно значительно улучшить, если за начальное паросочетание брать не пустое, а любое другое, полученное любым эвристическим методом.
\begin{lstlisting}[language=C++]
int n, k;
vector < vector<int> > g;
vector<int> mt;
vector<char> used;
 
bool try_kuhn (int v) {
	if (used[v])  return false;
	used[v] = true;
	for (size_t i=0; i<g[v].size(); ++i) {
		int to = g[v][i];
		if (mt[to] == -1 || try_kuhn (mt[to])) {
			mt[to] = v;
			return true;
		}
	}
	return false;
}
int main() {
	...  ...
 
	mt.assign (k, -1);
	vector<char> used1 (n);
	for (int i=0; i<n; ++i)
		for (size_t j=0; j<g[i].size(); ++j)
			if (mt[g[i][j]] == -1) {
				mt[g[i][j]] = i;
				used1[i] = true;
				break;
			}
	for (int i=0; i<n; ++i) {
		if (used1[i])  continue;
		used.assign (n, false);
		try_kuhn (i);
	}
 
	for (int i=0; i<k; ++i)
		if (mt[i] != -1)
			printf ("%d %d\n", mt[i]+1, i+1);
}
\end{lstlisting}
