\chapter{Графы}
Граф - множество вершин и ребер(заданных явно или не явно). Понятия используемые в дальнейшем:
\begin{itemize} 
\item 
    Ациклический граф
    \begin{mydef}
        Граф без циклов
    \end{mydef}
\item     
    Влентность/Степень вершины
    \begin{mydef}
        Количество ребер входящих/выходящих в вершину
    \end{mydef}
\item     
    Взвешенный граф
    \begin{mydef}
        Граф в котором у каждого ребра есть стоимость
    \end{mydef}
\item     
    Висячая вершина
    \begin{mydef}
        Вершина со степенью один
    \end{mydef}
\item     
    Гамильтонов путь
    \begin{mydef}
        Путь в графе содержащий каждую вершину ровно один раз
    \end{mydef}
\item     
    Гамильтонов цикл
    \begin{mydef}
        Цикл содержащий каждую вершину ровно один раз
    \end{mydef}
\item     
    Компонента связности
    \begin{mydef}
        Множество вершин и ребер графа такое, что из каждой его вершины
        достижима любая другая вершина этого множества
    \end{mydef}
\item     
    Компонента сильной связности
    \begin{mydef}
        Множество вершин и ребер ориентированного графа такое, что из каждой его           вершины достижима любая другая вершина этого множества
    \end{mydef}
\item     
    Кратные ребра
    \begin{mydef}
        Ребра связывающие одну и ту же пару вершин
    \end{mydef}
\item     
    Минимальный каркас
    \begin{mydef}
        Множество ребер соеденяющих все вершины графа без циклов и имеющее               минимальный суммарный вес
    \end{mydef}
\item     
    Паросочетания
    \begin{mydef}
        Множество попарно не смежных ребер
    \end{mydef}
\item     
    Точка сочленения
    \begin{mydef}
        Вершина после удаления которой количество компонент связности возрастает
    \end{mydef}
\item     
    Эйлеров путь
    \begin{mydef}
        Путь в графе содержащий каждое ребро ровно один раз
    \end{mydef}
\item     
    Эйлеров цикл
    \begin{mydef}
        Цикл содержащий каждое ребро ровно один раз
    \end{mydef}
\end{itemize}
