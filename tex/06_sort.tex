\chapter{Сортировки}
\section{Сортировка пузырьком $O(N^2)$}
Алгоритм состоит из повторяющихся проходов по сортируемому массиву. За каждый проход элементы последовательно сравниваются попарно и, если порядок в паре неверный, выполняется обмен элементов. \newline
Алгоритм \textbf{устойчив}

\begin{lstlisting}[language=C++]
#include <algorithm>
template<typename Iter>
void bubble_sort(Iter first, Iter last) {
    while(first < --last) {
        for(Iter it = first; it < last; it++) {
            if (*(it + 1) < *it) {
            	std::iter_swap(it, it + 1);
            }
		}
	}
}
\end{lstlisting}

\section{Сортировка выбором $O(N^2)$}
Алгоритм может быть \textbf{как устойчивым, так и неустойчивым}
Ниже представлены листинги \textbf{устойчивого} алгоритма:
\begin{enumerate}
\item Берём минимальный элемент в массиве
\item Вставляем значение этого элемента на первую неотсортированную позицию, при этом не меняя порядок остальных элементов
\item Сортируем "хвост" массива, убрав уже отсортированный промежуток
\end{enumerate}
\begin{lstlisting}[language=C++]
template <typename T, typename C = less<typename T::value_type> >
void select_sort(T first, T last, C c = C()) {
	if (first != last) {
        while (first != last - 1) {
            T mn = first;
            for (T i = first + 1; i != last; i++) {
                if (c(*i, *mn)) {
                    typename T::value_type tmp = *mn;
                    *mn = *i;
                    *i = tmp;
                }
            }
            first++;
        }
    }
}
\end{lstlisting}

\section{Сортировка вставками $O(N^2)$}
На каждом шаге алгоритма мы выбираем один из элементов входных данных и вставляем его на нужную позицию в уже отсортированном списке, до тех пор, пока набор входных данных не будет исчерпан.\newline
Алгоритм \textbf{стабилен}, особо эффективен при небольших входных данных (несколько десятков).
\begin{lstlisting}[language=C++]
template<class T> 
void InsertionSort(T *A, int N) {
    if (N < 2) 
    	return;
	for (int i = 1; i < N; i++)
    	for (int j = i; j && A[j] < A[j-1]; j--)
        	std::swap(A[j], A[j-1]);
}
\end{lstlisting}

\section{Сортировка слиянием $O(N\cdot \log{N})$}
В основе сортировки слиянием лежит принцип «разделяй и властвуй». Список разделяется на равные или практически равные части, каждая из которых сортируется отдельно. После чего уже упорядоченные части сливаются воедино. Алгоритм \textbf{стабилен}. Кушает $O(N) + O(\log{N})$ памяти. Ниже представлен листинг:
\begin{enumerate}
\item Массив рекурсивно разбивается пополам, если размер куска $|a| > 1$
\item Два единичных массива сливаются в общий результирующий массив, при этом из каждого выбирается меньший элемент (сортировка по возрастанию) и записывается в свободную левую ячейку результирующего массива. После чего из двух результирующих массивов собирается третий общий отсортированный массив, и так далее. В случае если один из массивов закончиться, элементы другого дописываются в собираемый массив
\item  Элементы перезаписываются из результирующего массива в исходный
\end{enumerate}

На следующей странице представлена реализация алгоритма:\newpage
\begin{lstlisting}[language=C++]
template<typename T>
void mergeSort(vector<T>& buf, size_t left, size_t right) {
    if(left >= right)
    	return;
  
 	size_t mid = (left + right) / 2;
    MergeSort(buf, left, mid);
    MergeSort(buf, mid + 1, right);
    merge(buf, left, right, mid);
}

template<typename T>
static void merge(vector<T>& buf, size_t left, size_t right, size_t mid) {
    if (left >= right || mid < left || mid > right) 
    	return;
    if (right == left + 1 && buf[left] > buf[right]) {
        swap(buf[left], buf[right]);
        return;
    }
 
    vector<T> tmp(&buf[left], &buf[left] + (right + 1));
    for (size_t i = left, j = 0, k = mid - left + 1; i <= right; i++) {
        if (j > mid - left) {      
            buf[i] = tmp[k++];
        } else if(k > right - left) {
            buf[i] = tmp[j++];
        } else {
            buf[i] = (tmp[j] < tmp[k]) ? tmp[j++] : tmp[k++];
        }
    }
}
\end{lstlisting}
.