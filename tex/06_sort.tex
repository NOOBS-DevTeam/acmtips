\chapter{Сортировки}
\section{Сортировка пузырьком $O(N^2)$}
Алгоритм состоит из повторяющихся проходов по сортируемому массиву. За каждый проход элементы последовательно сравниваются попарно и, если порядок в паре неверный, выполняется обмен элементов. \newline
Алгоритм \textbf{устойчив}

\begin{lstlisting}[language=C++]
#include <algorithm>
template< typename Iterator >
void bubble_sort(Iter first, Iter last)
{
    while(first < --last) {
        for(Iter it = first; it < last; it++) {
            if (*(it + 1) < *it) {
            	std::iter_swap(it, it + 1);
            }
		}
	}
}
\end{lstlisting}

\section{Сортировка выбором $O(N^2)$}
Алгоритм может быть \textbf{как устойчивым, так и неустойчивым}
Ниже представлены листинги \textbf{устойчивого} алгоритма:
\begin{enumerate}
\item Берём минимальный элемент в массиве
\item Вставляем значение этого элемента на первую неотсортированную позицию, при этом не меняя порядок остальных элементов
\item Сортируем "хвост" массива, убрав уже отсортированный промежуток
\end{enumerate}
\begin{lstlisting}[language=C++]
template <typename T, typename C = less<typename T::value_type> >
void select_sort(T first, T last, C c = C()) {
	if (first != last) {
        while (first != last - 1) {
            T mn = first;
            for (T i = first + 1; i != last; i++) {
                if (c(*i, *mn)) {
                    typename T::value_type tmp = *mn;
                    *mn = *i;
                    *i = tmp;
                }
            }
            first++;
        }
    }
}
\end{lstlisting}